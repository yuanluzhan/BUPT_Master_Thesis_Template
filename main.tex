%不需要区分奇偶页的请使用下面一行
\documentclass[a4paper,AutoFakeBold,oneside,12pt]{book}
%需要区分奇偶页的(即每一章第一页一定在奇数页上)请使用下面一行
%\documentclass[a4paper,AutoFakeBold,openright,12pt]{book}

\usepackage{BUPT_Master}
\usepackage{setspace}

\begin{document}

% 封面


% 内封(扉页)
\blankmatter
\includepdf[pages=-]{docs/cover.pdf}  

% 答辩委员会名单

% 声明
\blankmatter
\includepdf[pages=-]{docs/cover.pdf}  

% 摘要
% 中文摘要
\begin{titlepage}

    \pagestyle{empty}

    \begin{spacing}{1.05}
        \centering
        % 如果你的标题太长,可能会换行;如果你对换行位置不满意,请调节下面第一个{}的参数
        \parbox[c]{.75\textwidth}{\thesistitlefont{学位论文模板}}
    \end{spacing}

    \begin{spacing}{1.5}
        \centering
        \sanhao\quad{} \\ 
        \abszhname{摘\quad{}要} \\ 
    \end{spacing}
    \xiaosanhao\quad{}

    \normalsize

    中、英文摘要位于声明的次页,摘要应简明表达学位论文的内容要点,体现研究工作的核心思想。重点说明本项科研的目的和意义、研究方法、研究成果、结论,注意突出具有创新性的成果和新见解的部分。

    关键词是为文献标引工作而从论文中选取出来的、用以表示全文主题内容信息的术语。关键词排列在摘要内容的左下方,具体关键词之间以均匀间隔分开排列,无需其它符号。

    

    \quad{}

    \par\noindent\abszhkey{关键词}\quad{}%
    \abszhkeys{北京邮电大学\quad{}%
                      研究生\quad{}%
                      学位论文\quad{}%
                      模板\quad{}%
                      }%
\end{titlepage}

\thispagestyle{empty}

% Abstract
\begin{titlepage}

    \pagestyle{empty}

    \begin{spacing}{1.05}
        \centering
        % 如果你的标题太长,可能会换行;如果你对换行位置不满意,请调节下面第一个{}的参数
        \parbox[c]{.75\textwidth}{\thesistitleenfont{BUPT Master Thesis}}
    \end{spacing}

    \begin{spacing}{1.5}
        \centering
        \sanhao\quad{} \\ 
        \abszhname{ABSTRACT} \\ 
    \end{spacing}
    \xiaosanhao\quad{}
    \normalsize

    I have to admitted that all adminstration staffs in Beijing University of Posts and Telecommunications are bullshit.
    For bachelors, the thesis template provided by the adcdemic dean change every year and cotain some typos.
    For matsers, the graduate school even don't provide any template. 

    I have stayed in this university for almost six years, this university has seriously addicted to bureaucracy.
    The adminstration staffs are all bullshit, the only thing they do everyday is just eating and tortureing students as they can.
    \quad{}

    \par\noindent\absenkey{KEY WORDS}\quad{}%
    \absenkeys{Beijing University of Posts and Telecommunications\quad{}%
                      Master\quad{}%
                      Thesis\quad{}%
                      Template\quad{}%
                      }%
\end{titlepage}

\thispagestyle{empty}
% 目录

\fancypagestyle{plain}{\pagestyle{frontmatter}}

\fancypagestyle{plain}{\pagestyle{catalogmatter}}\pagenumbering{Roman}\tableofcontents % Content

% 符号说明

% 论文主体
\newpage\mainmatter
\fancypagestyle{plain}{\pagestyle{mainmatter}}
\chapter{示例}
论文主体是学位论文的核心部分,一般由理论分析、数据资料、计算方法、实验和测试方法,实验结果的分析和论证,个人的论点和研究成果,以及相关图表、照片和公式等部分构成。其写作形式可因科研项目的性质不同而变化,总体要求理论正确、逻辑清楚、层次分明、文字流畅、数据真实、公式推导计算结果无误。文中若有与导师或他人共同研究的成果,必须明确指出;如果引用他人的结论,必须明确注明出处,并与参考文献一致。


\section{算法}
这部分主要研究如何把大象放进冰箱,算法参照算法\ref{demo_algorithm}。
\begin{algorithm} 
	\begin{spacing}{1.3}
		\floatname{algorithm}{算法}
		\caption{如何把大象放进冰箱} 
		\label{demo_algorithm}
		\begin{algorithmic}[1] 
			\State 打开冰箱
            \State 把大象放进冰箱
            \State 关上冰箱
		\end{algorithmic}  
	\end{spacing}
\end{algorithm}
\section{表格}
推荐一个网站 https://www.tablesgenerator.com/
\section{公式}
推荐一个工具 mathpix
\section{其他文本示例}

\subsection{加粗}
如果你想加粗,请使用\textbf{textbf加粗}

\subsection{引用他人的话}

\begin{quotation}
    "Schools serve the same social functions as prisons and mental institutions- to define, classify, control, and regulate people."
        --Michel Foucalt
\end{quotation}


\subsection{列表}
\begin{itemize}
    \item 一个学位论文模板都没有的学校
    \item 你又能指望它干出什么正经事呢?
\end{itemize}
\subsection{脚注}
% 如果你的项目来源于科研项目,可以使用以下指令插入无编号脚注于正文第一页
\blfootnote{本项目来源于科研项目“基于\LaTeX{}的本科毕业设计”,项目编号0}

\chapter{实验结果}
不管你做的东西自己多么看不下去,实验结果都支持它很优秀这一结论。
% 参考文献

% 附录

% 致谢

% 作者攻读学位期间发表的学术论文目录



\end{document}